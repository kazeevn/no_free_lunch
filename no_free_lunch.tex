\documentclass[a4paper]{article}
\usepackage{polyglossia}
\usepackage{amssymb,amsmath,amsthm,amsfonts}
\usepackage{hyperref}
\usepackage[xindy]{glossaries}
\usepackage{lineno}
\linenumbers

\newtheorem{theorem}{Theorem}[section]
\newtheorem{corollary}{Corollary}[theorem]
\newtheorem{lemma}{Lemma}

\title{The Lack of A Priori Distinctions Between Learning Algorithms
  aka No Free Lunch Theorems for Learning} \author{Nikita Kazeev,
  based on David H. Wolpert}

% \newglossaryentry{OTS error}{
%   name=OTS error,
%   description={
%     \begin{equation}
%       P\left(q|d\right) = \frac{\delta(q \notin d_X)\pi(q)}{
%         \sum_q\left[\delta(q \notin d_X) \pi(q)\right]},
%     \end{equation}
%     where $\delta(z) \equiv 1$ if $z$ is true and $0$ otherwise.
%   }
% }

% \newglossaryentry{vertical}{
%   name=vertical,
%   description={
%     $P\left(d|f\right)$: iff $P\left(d|f\right)$ is independent
%     of the values $f(x,y_F)$ for $x \notin d_X$.
%   }
% }

% \newglossaryentry{$F$}{
%   name=$F$,
%   description={$f$ value space in}
% }

% \newglossaryentry{X}{
%   name=X,
%   description=dfdf%{The input set. Is finite,} %$|\mathbf{X}| = n \in \mathbb{N}$}
% }


\makeglossaries
\begin{document}
\maketitle
\begin{abstract}
  The objective of supervised learning is generalization,
  i. e. learning a relation and predicting on yet unseen data. In this
  paper we show that this problem can not be solved in general, for
  all target relations. If there are no restrictions on the structure
  of the problem, then for any two algorithms there are ``as many''
  targets on which each outperforms the other. This hold true even for
  random guessing.
\end{abstract}

\section{Introduction}
This is a streamlined and simplified representation of some results
from \textit{Wolpert, David H. "The lack of a priori distinctions between
learning algorithms." Neural computation 8.7 (1996): 1341-1390.}

The objective of supervised learning is generalization -- ``learning''
information about a process from a set of samples and then using it to
predict the outcome for examples yet unseen. It is widely advertised as an
assumption-free, ``data-driven'' approach, in contrast to explicit
statistical models -- see the famous paper by Breiman \cite{breiman}.

In this paper we recite several results obtained by David H. Wolpert,
that show the impossibility of a machine learning algorithm, that
would work for all targets. The paper does in no way argue that all
algorithms are equivalent \textit{in practice}. There are of course
algorithms that perform well over some classes of targets (often the
likes of what we see in the real life). But as we show here, for any
such algorithm there are many targets, at which it gets confused by
the data and preforms worse than \textit{random guessing}.

\section{Formalism}
Begin with two finite sets $\mathbf{X}$ and $\mathbf{Y}$. $\mathbf{X}$
is the input set, $\mathbf{Y}$ is the output set. Define a metric
(loss function): $L(y_1, y_2) \in \mathbb{R}$,
$y_1, y_2 \in \mathbf{Y}$. Introduce the target function
$f(x, y), x \in \mathbf{X}, y \in \mathbf{Y}$ -- an
$\mathbf{X}$-conditioned distribution over $\mathbf{Y}$. Select a
training set $d$ of $m$ $\mathbf{X}-\mathbf{Y}$ pairs, according to
some distribution $P(d|f)$. Select a test point
$q\in \mathbf{X}, q \notin d_X$ -- we are interested in the
generalization power. Such selection is called off the sample
(OTS). Take a classifier, train it on $d$, use it to predict on
$q$. Let $y_H$ be the prediction. Any classifier is completely
described by its behavior, $P(y_H|q,d)$. Also sample the target
distribution $f$ at point $q$, let $y_F$ be the result. Define loss
$c = L(y_H, y_F)$.

The results in the paper are various averages over $f$. $f$ is a set
of $|\mathbf{X}|\times |\mathbf{Y}|$ real numbers, so we can write a multidimensional
integral $\int A(f) df$ and average
$E_f A(f) = \int df A(f) / \int df 1$. All integrals over targets $f$
in this paper are implicitly restricted to the valid
$\mathbf{X}$-conditioned distributions over $\mathbf{Y}$. We do not
evaluate the integrals explicitly, but for the clarity sake, it is
worth to discuss them.

$\sum_y f(x,y) = 1$. Therefore, $f$ is a mapping from $\mathbf{X}$ to
an $|\mathbf{Y}|$-dimensional unit simplex. The integration volume $F$ is a
Cartesian product of unit simplices, which can be expressed using a
combination of Dirac delta functions and Heaviside step functions:

\begin{equation}
  \int_F A(f) df = \int A(f) df \prod_{i=0}^{|\mathbf{X}|}\left[\delta\left(\sum_{j=i\times |\mathbf{Y}|}^{(i+1)\times |\mathbf{Y}|-1}f_j - 1\right) \prod_{j=i\times |\mathbf{Y}|}^{(i+1)\times |\mathbf{Y}|-1}
    \theta\left(f_j\right)  \right].
\end{equation}

In this paper we consider \textit{homogeneous loss}, meaning that
\begin{equation}
  \exists \Lambda [\mathbb{R} \rightarrow \mathbb{R}]:
  \forall c \in \mathbb{R}, \forall y_H \in \mathbf{Y}:
  \sum_{y_F \in \mathbf{Y}} \delta\left[c, L(y_H, y_F)\right] = \Lambda(c),
\end{equation}
where $\delta$ is the Kronecker delta function.
Intuitively, such $L$ have no a priori preference for one $\mathbf{Y}$
value over another. For example, zero-one loss
($L(a,b) = 1 \text{ if } a\neq b, 0$ otherwise) is homogeneous, and
quadratic ($L(a,b) = (a - b)^2$; $a, b, \in \mathbb{R}$) is not. A
weaker version of No Free Lunch Theorem still holds for
non-homogeneous losses, they are discussed in \cite{Wolpert-yes-lunch}

Likelihood $P(d|f)$ determines how $d$ was generated from $f$. It is
\textit{vertical} if $P(d|f)$ is independent of the values $f(x, y_F)$
for $x \notin d_X$. For example, the conventional procedure, where $d$
is created by repeatedly choosing its $\mathbf{X}$ component by
sampling some distribution $\pi(x)$, and then choosing the associated
$d_Y$ value by sampling $f\left(d_x(i), y\right)$, results in a
vertical independent and identically distributed (IID) likelihood
\begin{equation}
  P(d|f) = \prod_{i=1}^m \pi(d_X(i)) f(d_X(i), d_Y(i)).
\end{equation}

\section{No Free Lunch}
The general idea behind the No Free Lunch theorems is calculating the
uniform average over $f$ of the distribution of classifier performance
(loss $c$) conditioned on various variables.

\subsection{Example}
Before writing the formal theorems, let us illustrate the
counter-intuitive idea of No Free Lunch on a simple example.

Take $\mathbf{X} = \left\{0,1,2,3,4\right\}$,
$\mathbf{Y} = \left\{0,1\right\}$, a uniform sampling distribution
$\pi(x)$, zero-one loss $L$.  For clarity we will consider only
deterministic $f$, i. e.
$f: \mathbf{X}\times \mathbf{Y} \rightarrow \{0,1\}$. Set the number of
distinct elements in the training set $m' = 4$. Let algorithm $A$
always predict the label most popular in the training set, algorithm
$B$ the least popular. In case the numbers of labels are equal, the
algorithms choose randomly.

Let $x_i \in \mathbf{X}$ be the feature vector and
$y_i \in \mathbf{Y}$ the label for $i$-th object. Let $c_A$ be the
loss on the test element for the algorithm $A$, $c_B$ for $B$. We show
that $E_f(c|f,m')$ is the same for $A$ and $B$.

\begin{enumerate}
\item There is only one $f$ for which for all $\mathbf{X}$ values,
  $\mathbf{Y}=0$. In this case algorithm $A$ works perfectly, $c_A=0$,
  algorithm $B$ always misses, $c_B=1$.
\item There are $C_5^1 = 5$ $f$s with only one $y = 1$, the rest being $0$. For
  each such $f$, the probability that the training set has all zeros
  is $0.2$. For these training sets, the true test label is $1$, $A$
  predicts $0$, $B$ predicts one, $c_A=1$, $c_B=0$. For the other 4
  training sets, $c_A=0$, $c_B=1$. Therefore, the expected value of
  $E c_A = 0.2\times 1 + 0.8 \times 0 = 0.2$ and
  $E c_B = 0.2\times 0 + 0.8 \times 1 = 0.8$
\item There are $C_5^2 = 10$ $f$s with two $y_i = 1$. There is a $0.4$
  probability, that the training set has one $1$. Therefore, the other
  $1$ is in the test set, and $c_A = 1$, $c_B = 0$. There is a $0.6$
  probability that the train set has two 1s. In that case both
  algorithms guess randomly and $E c_A = E c_B = 0.5$. So for each
  $f$, $E c_A = 0.4\times 1 + 0.6\times 0.5 = 0.7$,
  $E c_B = 0.4\times 0 + 0.6\times 0.5 = 0.3$. Note that $B$
  outperforms $A$.
\item The cases with three, four and five 1s are equivalent to the
  already described.
\item Averaging over $f$, we have
  $E_f c_A = \frac{1\times 0 + 5\times 0.2 + 10\times 0.7}{1+5+10} =
  0.5$,
  $E_f c_B = \frac{1\times 1 + 5\times 0.8 + 10\times 0.3}{1+5+10} =
  0.5$
\end{enumerate}

\subsection{Theorems}

\begin{lemma}
\begin{equation}
  P(c|d,f) = \sum_{y_H,y_F,q}\delta\left[c,L\left(y_H, y_F\right)\right] P\left(y_H|q, d\right)
  P\left(y_F| q, f\right)P\left(q|d\right)
\end{equation}
\label{lm:Pcdf}
\end{lemma}

\begin{proof}
\begin{equation}
  c = L\left(y_H, y_F\right)
\end{equation}

\begin{equation}
\begin{split}
  P\left(c|q,d,f\right)& = \sum_{y_H,y_F} \delta\left[c,L\left(y_H,
      y_F\right)\right] P\left(y_H, y_F| q,d,f\right) \\
  P\left(c|d,f\right) & = \sum_{y_H,y_F,q} \delta\left[c,L\left(y_H,
      y_F\right)\right] P\left(y_H, y_F|q, d,f\right) P\left(q|d\right) \\
  & = \sum_{y_H,y_F,q} \delta\left[c,L\left(y_H, y_F\right)\right]
  P\left(y_H|q,d\right) P\left(y_F|q,f\right) P\left(q|d\right)
\end{split}
\end{equation}
\end{proof}

\begin{theorem}
  For homogeneous loss $L$, the uniform average over all $f$ of
  $P\left(c|d,f\right)$ equals $\Lambda\left(c\right)/r$.
  \label{th:Pcdf}
\end{theorem}

For any fixed training set, for any OTS method $P(q|d)$ of selecting
the test point, including sampling the same $\pi(x)$, that was used to
select $d_X$, for any learning algorithm, any homogeneous loss $L$,
the average performance over all possible targets is a constant, that
only depends on $\left|\mathbf{Y}\right|$ and $L$.

This result ignores the relationship between $d$ and $f$ -- in other
words, the $\mathbf{Y}$ values for train and test sets are generated
from different distributions. Thus it is not particularly interesting
in itself, but will rather serve us a base for further inquiries.

\begin{proof}
  Using lemma \ref{lm:Pcdf}, the uniform average over all targets $f$
  of $P\left(c|d,f\right)$ can be written as
  \begin{equation}
    E_f \left[P\left(c|d,f\right)\right] = \sum_{y_H,y_F,q}
      \delta\left[c,L\left(y_H, y_F\right)\right]
      P\left(y_H|q,d\right) E_f \left[P\left(y_F|q,f\right)\right] P\left(q|d\right) \\
  \end{equation}

  \begin{equation}
    E_f \left[P\left(y_F|q,f\right)\right] = E_f f(q,y_F)
  \end{equation}
  Because $F$ is symmetric, the average is a constant that does
  not depend on $q$ and $y_F$. Also,
  \begin{equation}
    \sum_{y_F} E_f\left[f(q, y_F)\right] = E_f\left[\sum_{y_F} f(q, y_F)\right] = 1,  
  \end{equation}
  therefore
  \begin{equation}
    E_f\left[f(q, y_F)\right] = 1/r.
  \end{equation}
  Using the homogeneity property of $L$:
  \begin{equation}
    E_f P\left(c|d,f\right) = \sum_{y_H, q} \Lambda(c)
    P\left(y_H|q,d\right)  P\left(q|d\right) / r = \Lambda(c)/r
  \end{equation}
\end{proof}

\begin{theorem}
  For OTS error, a vertical $P\left(d|f\right)$, and a homogeneous
  loss $L$, the uniform average over all targets $f$ of
  $P\left(c|f, m\right) = \Lambda(c)/r$
  \label{th:Pcfm}
\end{theorem}

For any fixed training set size $m$, any vertical method of
training set generation, including the conventional IID-generated, for
any OTS method $P(q|d)$ of selecting the test point, including
sampling the same $\pi(x)$, that was used to select $d_X$, for any
learning algorithm, any homogeneous loss $L$, the average performance
over all possible targets is a constant, that only depends on
$\left|\mathbf{Y}\right|$ and $L$.

This is a valid No Free Lunch theorem, as advertised in the
beginning. If an algorithm ``beats'' some other, including the random
guess, on some $f$'s, it will necessary lose on the rest, so that the
averages would be the same.

\begin{proof}
  \begin{equation}
    P\left(c|f,m\right) = \sum_{d:|d|=m} P\left(c|d,f\right)P\left(d|f\right)
  \end{equation}

From \ref{lm:Pcdf}:
\begin{equation}
  P(c|d,f) = \sum_{y_H,y_F,q}\delta\left[c,L\left(y_H, y_F\right)\right] P\left(y_H|q, d\right)
  P\left(y_F| q, f\right)P\left(q|d\right).
\end{equation}
Because we consider OTS error, $P\left(q|d\right)$ will be non-zero
only for $q \notin d_X$, so $P\left(y_F| q, f\right)$ only depends on
components of $f(x,y)$ that correspond to $x\notin d_X$.

We also know that $P\left(d|f\right)$ is vertical, so it is
independent of the values $f(x,y_F)$ for $x \notin d_X$.

Therefore the integral can be split into two parts, over dimensions
corresponding to $d_X$ and $\mathbf{X} \setminus d_X$:
\begin{equation}
  E_f \left[P\left(c|f,m\right)\right] = \frac{\sum_{d:|d|=m}
    \left[\int df_{x\notin d_X} P\left(c|d,f \right)
      \int df_{x \in d_X} P\left(d|f\right)\right]}{\int df_{x\notin d_X} df_{x \in d_X} 1}
\end{equation}

Again using $P\left(c|d, f\right)$ independence from $x\in d_X$ and theorem \ref{th:Pcdf}:
\begin{equation}
  E_{f_{x\notin d_X}} \left[P\left(c|d, f\right)\right] = E_f \left[P\left(c|d, f\right)\right] = 
  \Lambda(c)/r
\end{equation}

\begin{equation}
  E_f \left[P\left(c|f,m\right)\right] = \Lambda(c)/r \frac{\sum_{d:|d|=m}
    \left[\int df_{x \in d_X} P\left(d|f\right)\right]}{\int df_{x \in d_X} 1} =
  \Lambda(c)/r
\end{equation}
\end{proof}

No Free Lunch theorem can also be formulated in Bayesian analysis
terms:
\begin{theorem}
  For OTS error, a vertical $P(d|f)$, uniform $P(f)$, and a
  homogeneous loss $L$, $P(c|d) = \Lambda(c)/r$.
  \label{Pcd}
\end{theorem}

\begin{proof}
  \begin{equation}
    E_f\left[P(c|d)\right] = \frac{\int df P(c|d,f) P(f|d)}{\int df}
  \end{equation}
  Using the Bayes theorem,
  \begin{equation}
    P(f) =  P(f|d) P(d) / P(d|f),
  \end{equation}
  and uniformity of $P(f)$:
  \begin{equation}
    E_f\left[P(c|d)\right] = \frac{\int df P(c|d,f) P(f) P(d|f) / P(d)}{\int df} = \alpha(d)
      \frac{\int df P(c|d,f) P(d|f)}{\int df 1},
  \end{equation}
  where $\alpha(d)$ is some function. Like in theorem \ref{th:Pcfm},
  the integral can be split into parts that depend on
  $f\left(x\in d_X\right)$ and $f\left(x\notin d_X\right)$:
  \begin{equation}
    E_f\left[P(c|d)\right] = \alpha(d) \frac{\int df_{x\notin d_X} P\left(c|d,f \right)
      \int df_{x \in d_X} P\left(d|f\right)}{\int df_{x\notin d_X} df_{x \in d_X} 1}.
  \end{equation}
  The integral $\int df_{x \in d_X} P\left(d|f\right)$ can again be
  absorbed into the $d$-dependent constant:
  \begin{equation}
    E_f\left[P(c|d)\right] = \beta(d) \frac{\int df_{x\notin d_X} P\left(c|d,f \right)
    }{\int df_{x\notin d_X} df_{x \in d_X} 1} = \frac{\Lambda(c)}{r} \frac{\beta(d)}{\int df_{x \in d_X}}.
    \label{eq:Pcd1}
  \end{equation}
  To find out the value of the constant, we integrate both sides by $c$:
  \begin{equation}
    \int dc E_f\left[P(c|d)\right] = 1 = \frac{\beta(d)}{\int df_{x \in d_X}}
    \int dc \frac{\Lambda(c)}{r}.
  \end{equation}
  From theorem \ref{th:Pcfm} we know that $\frac{\Lambda(c)}{r}$ is,
  in fact, a probability, thus $\int dc \frac{\Lambda(c)}{r} =
  1$. Therefore $\frac{\beta(d)}{\int df_{x \in d_X}} = 1$ as
  well. Substituting it back to \ref{eq:Pcd1}, we obtain:
  
  \begin{equation}
    E_f\left[P(c|d)\right] =  \frac{\Lambda(c)}{r}
  \end{equation}
\end{proof}

\section{Implications}
For learning theory, the No Free Lunch theorems invalidate any formal
performance guarantees, that do not make a restriction on the problem
class.

The NFL theorems also prove that performance on a ``test'' or
``validation'' set $T$, commonly used in practice, is not an agnostic
tool to compare algorithms. That is if we are interested in the error
for $x\notin \{d \cup T\}$, in absence of prior assumptions, the error
on $T$ is meaningless, no matter how many elements there are in $T$.

No Free Lunch theorems pose a philosophical paradox. The Wolpert's
paper begins with a quote from David Hume: ``Even after the
observation of the frequent conjunction of objects, we have no reason
to draw any inference concerning any object beyond those of which we
have had experience.'' All our experiences, the training set, belong
to the past. This includes any ``prior knowledge'' -- that targets
tend to be smooth, the Occams's razor, etc. The NFL theorems state,
that even if some knowledge and algorithms allowed you to generalize
well in the training set (the past), there are no formal guarantees
about its behavior in the future. So, if we are to subscribe to strict
empiricism and do not make any assumptions about the world, we must
accept that the world is unknowable. On the other hand, if we are to
claim the ability to predict the future, we must also admit that this
ability is based on some arbitrary assumptions. And there are
infinitely many possible assumptions and the choice can not be based
on anything empirical as otherwise it would fall under the prior
knowledge.

%\glsaddall
%\printglossaries

\begin{thebibliography}{99}
\bibitem{breiman}{Breiman, Leo. "Statistical modeling: The two
    cultures (with comments and a rejoinder by the author)."
    Statistical science 16.3 (2001): 199-231.}
\bibitem{Wolpert-yes-lunch} Wolpert, David H. "The existence of a
  priori distinctions between learning algorithms." Neural Computation
  8.7 (1996): 1391-1420.
\end{thebibliography}

\end{document}
